\documentclass[11pt,fleqn,twoside]{article}
\usepackage{makeidx}
\makeindex
\usepackage{palatino} %or {times} etc
\usepackage{plain} %bibliography style
\usepackage{amsmath} %math fonts - just in case
\usepackage{amsfonts} %math fonts
\usepackage{amssymb} %math fonts
\usepackage{lastpage} %for footer page numbers
\usepackage{fancyhdr} %header and footer package
\usepackage{mmpv2}
\usepackage{url}
\usepackage{hyperref}

% the following packages are used for citations - You only need to include one.
%
% Use the cite package if you are using the numeric style (e.g. IEEEannot).
% Use the natbib package if you are using the author-date style (e.g. authordate2annot).
% Only use one of these and comment out the other one.
\usepackage{cite}
%\usepackage{natbib}

\begin{document}

\name{Omar Ibrahim}
\userid{ibo1}
\projecttitle{Interactive Robot Chess}
\projecttitlememoir{Interactive Robot Chess} %same as the project title or abridged version for page header
\reporttitle{Project Outline}
\version{1.4}
\docstatus{Release} % change to Release when you are ready to submit your document
\modulecode{CS39440}
\degreeschemecode{G400}
\degreeschemename{Computer Science}
\supervisor{Maxim Buzdalov} % e.g. Neil Taylor
\supervisorid{mab168} % e.g. nst

%optional - comment out next line to use current date for the document
%\documentdate{8th February 2022}
\mmp

\setcounter{tocdepth}{4} %set required number of level in table of contents


%==============================================================================
\section{Project description}
%==============================================================================
The Interactive Robot Chess project will develop a robotic and desktop based application, to enable users to 
play chess against a robotic arm on a physical chess board. The chess engines that will be utilized for this project will be a 3rd party library. 
These libraries will used within a wrapper class, which enables a plug and play system for different chess engines.\newline

\noindent This project will develop a desktop application to allow users and developers to interact and obtain 
information from a robotic system. There will be a desktop-based interface, developed in Kotlin\cite{kotlin}, for users 
and developers. Utilizing Kotlin enables the usage of the interface in different desktop environments. 
This should have two different layouts, one for users and the other for developers. To determine which interface will be 
accessed, an authentication process will be used.\newline

\noindent The user interface should display the current board state to allow users to correct the system if the board 
state varies from the physical board state. Further, it should allow the user to choose from a variety of difficulties. 
Moreover, the user interface should include a visual and textual guide on how to setup the environment(calibration process) 
of the system to play a game of chess. The developer interface will be supporting the same functionality in addition to 
displaying additional backend information and the ability to change values within the system.\newline

\noindent The robotic application adds functionality to the robot arm and the overall system. This system will be 
developed in the ROS middleware environment\cite{ros}. The robotic application will consist of a physical robot arm for chess
movements, object detection and safety features; a simulation environment to test the functionality and safety of the 
system; a chess engine to enable the user to play a bot; an overhead camera to see the physical board state and enable us to implement
safety features; and a communication protocol to enable communication to the desktop application. The safety features will need to take 
physical objects, eg. human arm, into account. This ensures neither the robotic arm nor the user will be damaged or hurt.\newline 

%==============================================================================
\section{Proposed tasks}
%==============================================================================
The following tasks will be performed on this project:
\begin{itemize}
    \item \textbf{Investigation of the robotic arm and build process.} This task will explore the options
                  of robotic arms and how to implement them in the ROS middleware. Further it will be necessary 
                  to consider the connection to external devices for communication. Research into how to move 
                  chess pieces will be looked into.\newline
                  There will be a meeting with the Robotic group of the Computer Science department to 
                  determine which robotic arm will be used for this project based on its availability and the 
                  investigation done.\newline

    \item \textbf{Investigation of the chess engine and safety measures.} After confirming the robotic arm there 
                  will be a need to investigate which chess engine to use, an example for such a chess engine would 
                  be Stockfish \cite{stockfish}. This investigation also includes the need to discover the method of interacting with 
                  the chess engines.\newline
                  The safety measures will need to be determined based on law and above mentioned safety features. 
                  Further there is a need to research how to implement these safety measures based on the available 
                  equipment.

    \item \textbf{Investigation of object detection.} Here research to detect objects using cameras, such as 
                  the chess pieces, chess board and the users arm, will be accomplished. The detection will be 
                  realized by using libraries such as OpenCV \cite{openCV}.
                  
    \item \textbf{Development.} The development will consist of two main subtasks:
          \begin{itemize}
            \item \textbf{Robotic application.} The first part of the development will form the foundation of the 
                         project. It will construct the underlying data for the board. Arm movements to pick up and 
                         place chess pieces and getting to a resting position will be produced. This will also 
                         include the calibration and detection of above mentioned objects, to interact and gain 
                         information. Further the communication module and its corresponding protocol will generate 
                         the possibility to make internal information available to external devices. Safety measures 
                         are going to be implemented in the movement of the arm.
                         
            \item  \textbf{Desktop application.} The second part of the development will introduce the interface for 
                          the user and the developer. The interface is written in Kotlin to make it available to 
                          different desktop environments. It will include all above mentioned functionality. 
                          Further the communication to the robotic arm will be realized utilizing the communication 
                          protocol.
          \end{itemize}
    
    \item \textbf{Project Meetings and Project Workflow.} The project will involve weekly supervisor meetings. A 
                  project workflow will be setup up to ease the process of project management and realization of 
                  needed tasks. This will assist in the reports at the meetings and the writing of the project 
                  report.

    \item \textbf{Preparation for demonstrations.} There is a need to prepare for the Mid-Project and End-Project 
                  demonstration, using notes and creating a presentation.
                  
\end{itemize}


%==============================================================================
\section{Project deliverables}
%==============================================================================
The following project deliverables are expected.

\begin{itemize}
    \item \textbf{Mid-Project Demonstration Notes} - A set of project notes will be produced 
                  for the demonstration. This will be included as an appendix in the final 
                  report.

    \item \textbf{Desktop application} -- The desktop application as an executable file and 
                  possible installation manual. A version of this with its source code will be 
                  submitted for assessment. The corresponding source code will be available on a version 
                  control system.  

    \item \textbf{ROS Software} -- The robotic application code will include necessary files and 
                  third party scripts to run the application. An installation manual may also be 
                  added. A version of these files will be submitted for assessment and they 
                  will also be available on a version control system.

    \item \textbf{Test Results} -- This document will be a overview of test and experiment results. 
                  This is used to display all functionalities and safety features have been tested and 
                  achieved.

    \item \textbf{Final Report} -- This document will be the report and associated appendices. 
                  This document will also include all relevant acknowledgements for libraries, 
                  frameworks and tools used in this project.

    \item \textbf{Final Demonstration} -- Should include some prepared notes and a possible 
                  presentation.
\end{itemize}

\nocite{*} 

\addcontentsline{toc}{section}{Initial Annotated Bibliography}

% example of including an annotated bibliography. The current style is an author date one. If you want to change, comment out the line and uncomment the subsequent line. You should also modify the packages included at the top (see the notes earlier in the file) and then trash your aux files and re-run.
%\bibliographystyle{authordate2annot}
\bibliographystyle{IEEEannotU}
\renewcommand{\refname}{Annotated Bibliography}  % if you put text into the final {} on this line, you will get an extra title, e.g. References. This isn't necessary for the outline project specification.
\bibliography{References} % References file

\end{document}
