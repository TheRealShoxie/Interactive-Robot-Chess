\chapter{My Implementation}

\section{Introduction}
This chapter describes my implementation of the previous mentioned Scrum process.

\section{Scrum Team}
The Scrum Team will consist only of Omar Ibrahim(ibo1) and of Maxim Buzdalov(mab168) as an advisor.\newline

\subsection{Product Owner}
The Product Owner does not exist in this project. Thus the responsibility of this role will fall onto Omar Ibrahim. 
Maxim Buzdalov will be undertaking an advisory role and advising on decision.

\subsection{Scrum Master}
No Scrum Master will be deligated. This role is futile and thus will not be deployed.

\subsection{Developers}
Developers will consist of Omar Ibrahim. Maxim Buzdalov will be undertaking an advisory role for implementaion.


\section{Scrum Events}
This subsection describes the implementation of the Scrum Events.\newline

\subsection{The Sprint}
A Sprint will consist of one or two weeks. Starting on a Monday in that week and ending on the following Monday. \newline

\subsection{Sprint Planning}
Sprint Planning will occur every Monday. The start of the planning will begin in the meeting between Omar Ibrahim and 
Maxim Buzdalov where initial planning ideas will emerge. After that meeting more in-depth planning will occur and the 
Sprint Backlog will be defined for that Sprint. Initial Definition of Done will be created within the Sprint Planning.
Possible test cases will be defined within the Sprint Planning.

\subsection{Daily Scrum}
A Daily Scrum will be held at the end of each working day to get an overview of the progress of the Sprint.

\subsection{Sprint Review}
A Sprint Review will be held on a Sunday. This Sprint Review will be held at the start of the Monday meeting between 
Omar Ibrahim and Maxim Buzdalov.

\subsection{Sprint Retrospective}
A Sprint Restrospective will be held in the meeting between Omar Ibrahim and Maxim Buzdalov after the Sprint Review. 
This might not occur after every Sprint.

\section{Scrum Artifacts}
The Scrum Artifacts will be represented within software and tools.\newline

\subsection{Product Backlog}
The Product Backlog log will be represented as a list of User Stories inside the 
\href{https://github.com/TheRealShoxie/Interactive-Robot-Chess/issues}{\color{blue}Github Issues} with a label User Story.
A User Story\cite{user_story} is a general explanation of a software feature from the perspective of 
the end user. It is a way to define the value of a feature to the customer. A User Story will consist of 
As a ... wants to ..., so that .... Further it will consist of an estimation of time and a Definition of Done. 
The estimation of time is a how much time it will take for a story to be implemented. Definition of Done describes when 
the user can complete that task.


\subsection{Sprint Backlog}
Each Sprint Backlog will be represented as a Kanban board within a Github Project. Kanban board is a visualization tool 
for work\cite{kanban_board}. The Kanban board will consist of User Stories and Tasks or Spikes. Tasks and Spikes are 
a breakdown of a User Story into its Task or Spike. A Task is a software coding task that needs to be done. To be able 
to specify wether a task has been done, Unit Testing will be used. A Spike is research that needs to be done. 
The Kanban board will consist of following columns:

\begin{itemize}
    \item \textbf{Sprint Backlog --} where the user stories for this sprint will be held.
    \item \textbf{To Do(Tasks or Spikes) -- } where each task or spike for each user story within the Sprint Backlog 
                                              will be held that still needs to be done.
    \item \textbf{In Process(Tasks or Spikes) -- } where each task or spike that is being processed will be held.
    \item \textbf{Awaiting Review(Tasks or Spikes) -- } where each task or spike that is waiting for review will be held.
    \item \textbf{Done(Tasks or Spikes) -- } where each task or spike that is done and needs to be closed will be held.
\end{itemize}

\subsection{Increment}
The Increment will be the project software that will be held under the SRC folder within the 
\href{https://github.com/TheRealShoxie/Interactive-Robot-Chess}{\color{blue}Github Repository}.