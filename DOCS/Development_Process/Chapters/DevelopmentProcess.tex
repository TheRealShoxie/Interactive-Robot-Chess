\chapter{Development Process}

\section{Introduction}
This chapter describes the development process and initial definitions for that development process.

\section{Development Process}
\subsection{Agile Development}
An agile development process was chosen for this project based on following features:
\begin{itemize}
    \item Possibility to delivering working software frequently\cite{principles_agile}.
    \item Possibility to adapt to changing requirements, even late in development\cite{principles_agile}.
    \item Using working software as the primary measure for progress.\cite{principles_agile}
\end{itemize}

\subsection{Agile Process}
Scrum has been chosen as an agile development process. Scrum is popular and has no specified practices 
such as XP, which allows different approaches to testing\cite{example_processes}.

\subsection{Scrum}
Scrum\cite{scrum_guide} is an incomplete framework that helps people, teams and organizations manage a project. 
It allows the usage of various processes, techniques and methods that can be employed within the framework.
Scrum consists of the scrum team, scrum events and scrum artifacts. \newline

\subsubsection{Scrum Team}
Scrum Team consists of a product owner, scrum master and a developers\cite{scrum_guide}.\newline

\textbf{Product Owner}\newline
The product owner is responsible for developing and communicating the product goal, creating and communicating product 
backlog items, ordering these items and ensuring that the backlog is transparent, visible and understood
\cite{scrum_guide}.

\textbf{Scrum Master}\newline
The scrum master is accountable for ensuring the scrum team and product owner are following the scrum methodology by 
ensuring every member understands the theory and practice of scrum\cite{scrum_guide}.

\textbf{Developers}\newline
Developers are are the people on the team that create work together to create the product\cite{what_is_scrum}.

\subsubsection{Scrum Events}
Scrum Events\cite{scrum_guide} are used to create regularity and to minimize meetings. These events are designed to 
enable the required transparency and the formal opportunity to inspect and adapt scrum artifacts. Scrum events 
consist of the sprint, sprint planning, daily scrum, sprint review and a sprint retrospective.

\textbf{The Sprint}\newline
Scrum sprints\cite{what_is_scrum} are short cycles of one month or less, which is used to get work done. The sprint 
contains all other scrum events and it starts at the conclusion of a previous sprint.

\textbf{Sprint Planning}\newline
Sprint planning\cite{what_is_scrum} is the event at the start of a sprint to plan out the work that will be done 
during the sprint

\textbf{Daily Scrum}\newline
Daily Scrum\cite{what_is_scrum} is an event held every day of the sprint to inspect progress towards the sprint goal 
that was defined in the sprint planning event.

\textbf{Sprint Review}\newline
Sprint review\cite{what_is_scrum} is an event at the end of the sprint where the team review accomplishments in the 
sprint, what environment changes were recorded and what to do next.

\textbf{Sprint Retrospective}\newline
Sprint retrospective\cite{what_is_scrum} is used to analyze the last sprint how it went and to identify the most 
helpful changes to improve effectiveness.

\subsubsection{Scrum Artifacts}
Scrum Artifacts\cite{scrum_guide} represent work and values of the product. Each artifact consists of a commitment 
and focus against which progress can be measured. The three types of scrum artifacts are:

\begin{itemize}
    \item Product Backlog and its Product Goal
    \item Sprint Backlog and its Sprint Goal
    \item Increment and its Definition of Done
\end{itemize}

\textbf{Product Backlog}\newline
The Product Backlog\cite{scrum_guide} is an emergent, ordered list of items what are needed to improve the product. 
Items which are deemed Done can be selected for one sprint. The commitment for the Product Backlog is the Product 
Goal which describes a future state of the product. It is the objective for the scrum team.

\textbf{Sprint Backlog}\newline
The Sprint Backlog\cite{what_is_scrum} is an evolving visible list of work that the Developer's plan for the Sprint. 
The commitment for the Sprint Backlog is the Sprint Goal\cite{scrum_guide} which is the single objective for the Sprint. 

\textbf{Increment}\newline
The Increment\cite{scrum_guide} is small pieces of work that are concrete stepping stones towards the Product Goal. Each 
Increment is supplement to all prior Increments and thoroughly verified. This ensures that all Increments work together. 
The commitment for the increment is the Definition of Done\cite{scrum_guide} which is a description of the state 
of the Increment when it meets the quality measures required. If a Product Backlog item meets the Definition of Done, an 
Increment is born. 